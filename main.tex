\documentclass[12pt, a4paper]{article}
\usepackage[T1]{fontenc}

\usepackage[english]{babel}
\usepackage{microtype}
\usepackage{amsmath,amsfonts,amsthm}
\usepackage{graphicx}
\usepackage{url}
\usepackage{geometry}
\usepackage{hyperref}
\usepackage{fancyhdr}
\usepackage{enumitem}
\usepackage{tabularx}
\usepackage{mathtools}
\usepackage{csquotes}
\usepackage[style=apa]{biblatex}
\addbibresource{ref.bib}
% Adjust margins here

\geometry{left=3cm, right=3cm, top=3cm, bottom=3cm, headheight=15pt}
\addtolength{\topmargin}{-2.5pt}
% Increase bottom margin to lower page numbering

\pagestyle{fancy}
\fancyhf{} % clear all header and footer fields
\fancyhead[L]{MATH 323: Actuarial Mathematics I  } % left header
\fancyhead[R]{Homework Report 3} % right header
\fancyfoot[C]{\thepage} % center footer
\renewcommand{\headrulewidth}{0.4pt} % header rule width
\renewcommand{\footrulewidth}{0.4pt} % footer rule width

\begin{document}

\begin{titlepage}
    \centering
    
    \vspace*{0.5cm}
    
    {\Large\bfseries MATH 323: Actuarial Mathematics I\par}
    
    \vspace{1cm}
    
    {\large Homework Report 3\par}
    
    \vspace{0.5cm}
    
    {\today\par}
    
    \vspace{1pt}
    
    \includegraphics[width=0.3\textwidth]{NU-logo.png}\\
    \includegraphics[width=0.15\textwidth]{sosah-logo.png}

    \vspace{0.5cm}
    
    Submitted for {\bf MATH 323: Actuarial Mathematics I} at the School of Sciences and Humanities, Department of Mathematics, Nazarbayev University
    
    \vspace{0.5cm}
    
    {\large Student Name:\par}
    \begin{itemize}[leftmargin=5cm,rightmargin=4cm]
        \item  Aigerim Tursynbekkyzy - ID: 202043550
    \end{itemize}

    \vspace{0.5cm}
    
    \flushleft{
  Subject Area: {\bf Theory of Interest} \\
  Description: {\bf Homework Problems in Chapter 6 and Chapter 7 \\
  Course Instructor : {\bf Dongming Wei} \\
    }
    
    \vspace{0.5cm}
    
    {\footnotesize In submitting this work we are indicating
    that we have read the University's Academic Integrity Policy. We
    declare that all material in this assessment is our own work except
    where there is clear acknowledgment and reference to the work of
    others.\par}
\end{titlepage}
The following is the used as solutions samples for each problem:
\newpage
\section*{Problems}

\subsection*{Question 1, Section 2, Chapter 6  (\cite{toi3rd})}

\noindent Find the price which should be paid for a zero coupon bond that matures for \$1000 in 10 years to yield:
   (a) 10\% effective.
   
   (b) 9\% effective.
   
   (c) Thus, a 10\% reduction in the yield rate causes the price to increase by what percentage?


\subsection*{Question 3, Section 2, Chapter 6  (\cite{toi3rd})}

\noindent A 26--week T-bill is bought for \$9600 at issue and will mature for \$10000. Find the yield rate computed as:
   
   (a) A discount rate, using the typical method for counting days on a T-bill.
   
   (b) An annual effective rate of interest, assuming the investment period is exactly half a year.

   \subsection*{Question 5, Section 3, Chapter 6  (\cite{toi3rd})}

\noindent Two \$1000 bonds redeemable at par at the end of the same period are bought to yield 4\% convertible semiannually. 
   One bond costs \$1136.78 and has a coupon rate of 5\% payable semiannually. 
   The other bond has a coupon rate of 2\tfrac{1}{2}\% payable semiannually. 
   Find the price of the second bond.

   \subsection*{Question 8, Section 3, Chapter 6  (\cite{toi3rd})}

\noindent An investor owns a \$1000 par value 10\% bond with semiannual coupons. The bond will mature at par at the end of 10 years.
   The investor decides that an 8-year bond would be preferable. Current yield rates are 7\% convertible semiannually.
   The investor uses the proceeds from the sale of the 10\% bond to purchase a 6\% bond with semiannual coupons, 
   maturing at par at the end of 8 years. Find the par value of the 8-year bond. Answer to the nearest dollar.

    \subsection*{Question 10, Section 3, Chapter 6  (\cite{toi3rd})}

\noindent For the bond in Example 6.3, determine:

    (a) Nominal yield based on the par value.
    
    (b) Nominal yield based on the redemption value.
    
    (c) Current yield.
    
    (d) Yield to maturity.

    \subsection*{Question 11, Section 4, Chapter 6  (\cite{toi3rd})}

\noindent For a \$1 bond the coupon rate is 150\% of the yield rate and the premium is \(p\).
    For another \$1 bond with the same number of coupons and the same yield rate, 
    the coupon rate is 75\% of the yield rate. Express the price of the second bond as a function of \(p\).

\subsection*{Question 13, Section 4, Chapter 6  (\cite{toi3rd})}

\noindent A 10-year bond with semiannual coupons is bought at a discount to yield 9\% convertible semiannually.
    If the amount for accumulation of discount in the next-to-last coupon is \$8, 
    find the total amount for accumulation of discount during the first four years in the bond amortization schedule.

   \subsection*{Question 14, Section 4, Chapter 6  (\cite{toi3rd})}

\noindent A \$1000 par value five-year bond with a coupon rate of 10\% payable semiannually and redeemable at par 
    is bought to yield 12\% convertible semiannually. 
    Find the total of the interest paid column in the bond amortization schedule.

    \subsection*{Question 16, Section 4, Chapter 6  (\cite{toi3rd})}

\noindent (a) Find the book values for the bond in Table 6.2 by the straight line method.
    
    (b) Find the book values for the bond in Table 6.3 by the straight line method.
    
    (c) What can you conclude from comparing the answers in (a) and (b) with the true values from Tables 6.2 and 6.3?

       \subsection*{Question 18, Section 5, Chapter 6  (\cite{toi3rd})}

\noindent Find the flat price, accrued interest, and market price (book value) two months after purchase 
    for the bond in Table 6.3. Use all three methods.

    \subsection*{Question 19, Section 5, Chapter 6  (\cite{toi3rd})}

\noindent A \$1000 bond with semiannual coupons at \(i^{(2)} = 6\%\) matures at par on October 15, \(2t+15\).
    The bond is purchased on June 28, \(2t\), to yield \(i^{(2)} = 7\%\).
    Find the purchase price. Assume simple interest between coupon dates and use exact day count.

    \subsection*{Question 20, Section 6, Chapter 6  (\cite{toi3rd})}

\noindent A \$100 par value 12-year bond with 10\% semiannual coupons is selling for \$110. 
    Find the yield rate convertible semiannually:
    (a) Using the exact method.
    (b) Using the refined bond-salesman method.

    \subsection*{Question 21, Section 6, Chapter 6  (\cite{toi3rd})}

\noindent  An investor buys two 20-year bonds, each having semiannual coupons and each maturing at par.
    One bond has par value \$500 and coupon \$45; the other has par value \$1000 and coupon \$30.
    The dollar amount of premium on the first bond is twice the dollar amount of discount on the second.
    Find the yield rate convertible semiannually.

    \subsection*{Question 23, Section 6, Chapter 6  (\cite{toi3rd})}

\noindent An \(n\)-year \$1000 par value bond with 4.20\% coupons is purchased to yield an annual effective rate \(i\).
    Given:
    (a) If the coupon rate were 5.25\%, the price would increase by \$100.
    (b) The PV of all coupon payments equals the PV of redemption of \$1000.
    Calculate \(i\).

    \subsection*{Question 24, Section 7, Chapter 6  (\cite{toi3rd})}

\noindent A \$1000 par value bond has 8\% semiannual coupons and is callable at par at the end of years 10--15.
    (a) Find the price to yield 6\% convertible semiannually.
    (b) Find the price to yield 10\% convertible semiannually.
    (c) If the bond in (b) is actually called at year 10, find the yield rate.
    (d) If the bond is puttable instead of callable, rework (b).

    \subsection*{Question 25, Section 7, Chapter 6  (\cite{toi3rd})}

\noindent A \$1000 par value 8\% bond with quarterly coupons is callable five years after issue.
    The bond matures in 10 years and is sold to yield a nominal 6\% convertible quarterly,
    assuming it will not be called. Find the redemption value at year 5 that preserves this yield. 
    Round to the nearest dollar.

    \subsection*{Question 26, Section 7, Chapter 6  (\cite{toi3rd})}

\noindent A \$1000 par value 4\% semiannual coupon bond matures in 10 years.
    Callable at \$1050 in years 4--6, at \$1025 in years 7--9, and at \$1000 in year 10.
    Find the maximum price that guarantees a yield of 5\% convertible semiannually.

    \subsection*{Question 27, Section 7, Chapter 6  (\cite{toi3rd})}

\noindent A \$1000 par, 9\% semiannual coupon bond was called for \$1100 before maturity.
    It was bought for \$918 immediately after a coupon payment and held to call. 
    The nominal yield rate convertible semiannually was 10\%.
    Calculate the number of years held. Round to nearest integer.

    \subsection*{Question 28, Section 7, Chapter 6  (\cite{toi3rd})}

\noindent A \$1000 par value bond pays annual coupons of \$80. Redeemable at par in 30 years; callable after year 10 at \$1050.
    Based on the desired yield rate, two purchase prices are:
    (i) If called at year 10: \$957.
    (ii) If held to maturity: \$897.
    The investor buys at the highest price guaranteeing the yield regardless of call time.
    The bond is called after 20 years. Find the annual yield rate earned.

    \subsection*{Question 1, Section 2, Chapter 7  (\cite{toi3rd})}

\noindent A ten-year investment requires \$100{,}000 initially and annual maintenance beginning at end of year.
    Maintenance cost is \$3000 first year, increasing 6\% annually. Returns are \$30{,}000 first year, decreasing 4\% each year.
    Find \(i_{0.08}\) to the nearest dollar.

\subsection*{Question 2, Section 2, Chapter 7  (\cite{toi3rd})}

\noindent An investor contributes \$7000 now and \$1000 at year 2 in exchange for \$4000 at year 1 and \$5500 at year 3.
    Find:
    (a) \(P(0.09)\)
    (b) \(P(1.10)\)

\subsection*{Question 4, Section 2, Chapter 7  (\cite{toi3rd})}

\noindent  ABC Manufacturing builds a plant costing \$2{,}000{,}000 now with 10-year life.
   At end of year 5 renovation cost \$X is required.
   Returns: \$300{,}000 yearly for 5 years, then \$600{,}000 for 5 years.
   Find the maximum \(X\) that yields at least 12\%.

\subsection*{Question 5, Section 2, Chapter 7  (\cite{toi3rd})}

\noindent Project P: invest \$4000 at time 0, receive \$2000 at time 1 and \$4000 at time 2.
   Project Q: invest \$X at time 2, receive \$2000 at time 0 and \$4000 at time 1.
   Using 10\% effective rate, determine \(X\) such that NPVs are equal.

\subsection*{Question 6, Section 2, Chapter 7  (\cite{toi3rd})}

\noindent The ABC Real Estate Development Corp. has cash flows (in millions) for years 0--10:

   Time: 0 1 2 3 4 5 6 7 8 9 10  
   NCF: -7.9, 1.4, 1.1, 1.0, 1.0, 1.0, 0.91, 0.9, 0.9, 0.45, 10.0

   (a) Compute NPV at 15\%.  
   (b) Compute the IRR.

\subsection*{Question 8, Section 3, Chapter 7  (\cite{toi3rd})}

\noindent Payments of \$100 now and \$108.15 in two years are equivalent to \$208 in one year at rates \(i\) and \(j\).
   Find \(|i - j|\).

\subsection*{Question 9, Section 3, Chapter 7  (\cite{toi3rd})}

\noindent A project has cash flows:
   Time 0 = 1000, Time 1 = A, Time 2 = B.
   Determine A and B such that the project yields both 20\% and 40\%.

\subsection*{Question 10, Section 4, Chapter 7  (\cite{toi3rd})}

\noindent An investor deposits \$10{,}000 into Fund A for 10 years at 6\% effective.
    Years 1--5: interest reinvested in Fund B at 4\%.  
    Years 6--10: interest reinvested in Fund C at 5\%.  
    (a) Find the total accumulated value.  
    (b) Find the overall yield rate.

\subsection*{Question 11, Section 4, Chapter 7  (\cite{toi3rd})}

\noindent Accumulate \$1000 in 10 years by deposits at beginning of each year.
    Deposits earn 8\% but interest must be reinvested at 4\%.
    Compute required deposit.

\subsection*{Question 13, Section 4, Chapter 7  (\cite{toi3rd})}

\noindent An investor receives \$1000 at end of each year for 5 years.
    Each payment earns 4\% until year-end; then interest reinvested at 3\%.
    Find purchase price to yield 4\%.

\subsection*{Question 15, Section 4, Chapter 7  (\cite{toi3rd})}

\noindent A buys a 10-year \$1000 6\% semiannual coupon bond. Each coupon is invested at 4\% effective.
    After 10 years A has earned 7\% annual effective yield. Compute \(i\).

\subsection*{Question 17, Section 4, Chapter 7  (\cite{toi3rd})}

\noindent A new partnership borrows \$25{,}000 at 8\% amortized over 4 years. After 1 year it refinances with Lender \#2.
    Lender \#1 reinvests proceeds at 8\% for 3 more years. Find Lender \#1’s yield rate over 4 years.

\subsection*{Question 18, Section 4, Chapter 7  (\cite{toi3rd})}

\noindent An investor pays \$100{,}000 today for a 4-year investment producing \$50{,}000 at ends of years 2,3,4.
    Each cash flow reinvested at 8\%. Compute \(P(i)\), the NPV at 10\%.

\subsection*{Question 20, Section 5, Chapter 7  (\cite{toi3rd})}

\noindent An account earns 6\% with initial balance \$10{,}000. Deposits: \$1800 at month 2, \$900 at month 8. 
    Withdrawal: \$5000 at month 6. Year-end balance: \$10{,}636.
    Using simple interest approximation, determine \(K\).

\subsection*{Question 21, Section 5, Chapter 7  (\cite{toi3rd})}

\noindent An insurance company earned simple interest of 8\% based on:

   Assets at start: 25{,}000{,}000  
   Sales revenue: X  
   Net investment income: 2{,}000{,}000  
   Salaries: 2{,}000{,}000  
   Other expenses: 750{,}000  

   All flows occur mid-year. Find effective yield rate.

\subsection*{Question 25, Section 6, Chapter 7  (\cite{toi3rd})}

\noindent Deposits of \$1000 at times 0 and 1. Fund balances: \$1200 at time 1, \$2200 at time 2.
    (a) Compute dollar-weighted annual yield.
    (b) Compute time-weighted annual yield.

\subsection*{Question 26, Section 6, Chapter 7  (\cite{toi3rd})}

\noindent Invest \$2000 at t=0 and \$1000 at t=1/2. Value at t=1 is \$2300.
    Find the amount needed at t=1/2 for the time-weighted return to exceed the dollar-weighted return by 0.02.

\subsection*{Question 27, Section 6, Chapter 7  (\cite{toi3rd})}

\noindent Invest \$2000 at t=0 and \$1000 at t=1/2. Account has \$2100 at t=1/2 and \$3213.60 at t=1.
    (a) Find dollar-weighted return.
    (b) Find time-weighted return.

\subsection*{Question 28, Section 6, Chapter 7  (\cite{toi3rd})}

\noindent Investor deposits D on Jan 1. Summary:

   March 15: value 40, deposit 20  
   June 1: value 80, deposit 80  
   Oct 1: value 175, deposit 75  

   June 30 value: 157.50  
   Dec 31 value: X  

   Using time-weighted method where first 6-month yield equals annual time-weighted yield for full year, compute X.

\subsection*{Question 30, Section 6, Chapter 7  (\cite{toi3rd})}

\noindent Let A = Jan 1 balance, B = June 30 balance, C = Dec 31 balance.
    (a) Show dollar-weighted = time-weighted yield if no deposits/withdrawals.
    (b) With deposit D just after June 30, derive yield expressions.
    (c) Rework if deposit just before June 30.
    (d) Interpret equality of dollar-weighted yields from (b) and (c).
    (e) Show time-weighted yield in (b) is greater than in (c).

\subsection*{Question 32, Section 7, Chapter 7  (\cite{toi3rd})}

\noindent Find \( \frac{s_3}{i} \) from Table 7.2 assuming the first payment is made in calendar year \(z+3\).

\subsection*{Question 33, Section 7, Chapter 7  (\cite{toi3rd})}

\noindent A person deposits \$1000 on Jan 1, year \(z+6\), earning rates in Table 7.2.
    Let accumulated values on Jan 1, \(z+9\) be:
    P = investment-year method,  
    Q = portfolio-yield method,  
    R = reinvest-at-new-rate method.  
    Determine P, Q, and R.

\subsection*{Question 35, Section 7, Chapter 7  (\cite{toi3rd})}

\noindent Given \(1 + i_t = (1.08 + 0.005)^{1+0.1t}\) for \(t = 1,\dots,10\),
    find the equivalent level effective rate for investing \$1000 for 3 years beginning in year 5.

\newpage

\section*{Solutions}

\subsection*{Question 1, Section 2, Chapter 6  (\cite{toi3rd})}


\newpage
\printbibliography
\end{document}
