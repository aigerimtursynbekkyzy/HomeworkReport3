\documentclass[12pt, a4paper]{article}
\usepackage[T1]{fontenc}

\usepackage[english]{babel}
\usepackage{microtype}
\usepackage{amsmath,amsfonts,amsthm}
\usepackage{graphicx}
\usepackage{url}
\usepackage{geometry}
\usepackage{hyperref}
\usepackage{fancyhdr}
\usepackage{enumitem}
\usepackage{tabularx}
\usepackage{mathtools}
\usepackage{csquotes}
\usepackage[style=apa]{biblatex}
\addbibresource{ref.bib}
% Adjust margins here

\geometry{left=3cm, right=3cm, top=3cm, bottom=3cm, headheight=15pt}
\addtolength{\topmargin}{-2.5pt}
% Increase bottom margin to lower page numbering

\pagestyle{fancy}
\fancyhf{} % clear all header and footer fields
\fancyhead[L]{MATH 323: Actuarial Mathematics I  } % left header
\fancyhead[R]{Homework Report #1} % right header
\fancyfoot[C]{\thepage} % center footer
\renewcommand{\headrulewidth}{0.4pt} % header rule width
\renewcommand{\footrulewidth}{0.4pt} % footer rule width

\begin{document}

\begin{titlepage}
    \centering
    
    \vspace*{0.5cm}
    
    {\Large\bfseries MATH 323: Actuarial Mathematics I\par}
    
    \vspace{1cm}
    
    {\large Homework Report #1\par}
    
    \vspace{0.5cm}
    
    {\today\par}
    
    \vspace{1pt}
    
    \includegraphics[width=0.3\textwidth]{NU-logo.png}\\
    \includegraphics[width=0.15\textwidth]{sosah-logo.png}

    \vspace{0.5cm}
    
    Submitted for {\bf MATH 323: Actuarial Mathematics I} at the School of Sciences and Humanities, Department of Mathematics, Nazarbayev University
    
    \vspace{0.5cm}
    
    {\large Student Name:\par}
    \begin{itemize}[leftmargin=5cm,rightmargin=4cm]
        \item  xxxxxxxxxxx - ID: 202031896
    \end{itemize}

    \vspace{0.5cm}
    
    \flushleft{
  Subject Area: {\bf Theory of Interest} \\
  Description: {\bf Homework Problems in Chapter 1 and Chapter 22 \\
  Course Instructor : {\bf Dongming Wei} \\
    }
    
    \vspace{0.5cm}
    
    {\footnotesize In submitting this work we are indicating
    that we have read the University's Academic Integrity Policy. We
    declare that all material in this assessment is our own work except
    where there is clear acknowledgment and reference to the work of
    others.\par}
\end{titlepage}
The following is the used as solutions samples for each problem:
\newpage
\section*{Problems}
\subsection*{Question 1, Section 2, Chapter 2  (\cite{boyce11th})}

\noindent\textbf{Assigned to Dina Kalibekova}\\

\noindent Five cards are dealt from a standard 52-card deck. What is the probability that the sum of the faces on the five cards is 48 or more?


\subsection*{Question 2, Section 2, Chapter 2 (\cite{boyce11th})}

\noindent\textbf{Assigned to Zhannur Kazenov}\\

\noindent Dana is not the world’s best poker player. Dealt a 2 of diamonds, an 8 of diamonds, an ace of hearts, an ace of clubs, and an ace of spades, she discards the three aces. What are her chances of drawing to a flush?

\subsection*{Question 3, Section 2, Chapter 2 (\cite{boyce11th})}

\noindent\textbf{Assigned to Alina Abdrakhmanova}\\

\noindent Tim is dealt a 4 of clubs, a 6 of hearts, an 8 of hearts, a 9 of hearts, and a king of diamonds. He discards the 4 and the king. What are his chances of drawing to a straight flush? To a flush?


\subsection*{Question 4, Section 2, Chapter 2 (\cite{boyce11th})}

\noindent\textbf{Assigned to Akbota Assainova}\\

A poker player is dealt a 7 of diamonds, a queen of diamonds, a queen of hearts, a queen of clubs, and an ace of hearts. He discards the 7. What is his probability of drawing to either a full house or four-of-a-kind?


\subsection*{Question 5, Section 2, Chapter 2 (\cite{boyce11th})}

\noindent\textbf{Assigned to Dariya Kalymova}\\

\noindent A bridge hand (thirteen cards) is dealt from a stan dard 52-card deck. Let A be the event that the hand contains four aces; let B be the event that the hand contains four kings. Find
\begin{math}
 P(A \cup B).
 \end{math}\par 

\newpage

\section*{Solutions}

\subsection*{Question 1, Section 2, Chapter 2  (\cite{boyce11th})}

\noindent\textbf{Assigned to Dina Kalibekova}\\

\noindent The number of ways that five cards are dealt from a standard 52-card deck is:
$$
    C=\frac{52!}{5!(52-5)!}= 2598960
$$
4 cards which contain the number 10 and one card which contains 9:
$$
    (1\times4)+(4\times6)+(4\times1)=32
$$
The probability that the sum of the faces on the five cards is 48 or more is:
$$
   P(Sum \ \ is \ \ 48 \ \ or \ \ more) = \frac{m}{n} = \frac{32}{2598960},$$
\newline
where m - The number of ways of selecting the sum of the faces and n - Total number of ways
\newline
Therefore, the probability that the sum of the faces on the five cards is 48 or more is:

    $$\frac{32}{2598960}$$

\subsection*{Question 2, Section 2, Chapter 2 (\cite{boyce11th})}

\noindent\textbf{Assigned to Zhannur Kazenov}\\

\noindent To get a flush, Dana needs to draw any 3 of the remaining 11 diamonds. Since only 47 cards are effectively left in the deck (others may already have been dealt, but their identities are unknown), 
\newline P(Dana draws to flush) = P(A)
\[
    P(A) =\frac{\binom{11}{3}}{\binom{47}{3}}=0.0101758
\]
\subsection*{Question 3, Section 2, Chapter 2 (\cite{boyce11th})}

\noindent\textbf{Assigned to Alina Abdrakhmanova}\\

\noindent Tim is dealt a 4 of a clubs, 8 of hearts, 9 of hearts, and king of diamonds. So, he has been dealt 5 cards and there are 47 cards remaining in the deck. To get a straight flush, Tim needs to have all the cards of the same suit and in sequence. He has 3 hearts in hand that can be made into sequence if 7 of hearts is added to it. Then another 1 card is needed (either 5 or 10 of hearts). \newline
7 of hearts can be drawn in \begin{equation}
    \binom{1}{1}
\end{equation} ways. \newline
Any of 5 or 10 of hearts can be drawn in \begin{equation}
    \binom{2}{1}
\end{equation} ways. \newline
Let A be an event that a straight flush is drawn. \newline
Then probability of it happening is: \begin{equation}
    P(A) =\frac{\binom{1}{1}\binom{2}{1}}{\binom{47}{2}}
\end{equation} \newline
To get a flush, which is when all cards are of the same suit and not necessarily in sequence, Tim needs to choose 2 more cards from of hearts. He can do it in $\binom{10}{2}$ ways. It includes 2 cards that will make a sequence a \textbf{straight flush}. So total number of ways to get a flush is $\binom{10}{2}-2$. \newline
Let B be an event that flush is drawn.
Then probability of it happening is: \begin{equation}
    P(A) =\frac{\binom{10}{2}-2}{\binom{47}{2}}
\end{equation}
\subsection*{Question 4, Section 2, Chapter 2 (\cite{boyce11th})}

\noindent\textbf{Assigned to Akbota Assainova}\\

\noindent Firstly, let's discuss about the cases of full house or four-of-a-kind to solve this problem. 
\newline 
\textbf{Four of kind:} 
a hand where four of the cards are with the same rank and another one with different rank. \textit{Example:} $K\varheartsuit, K\spadesuit, K\clubsuit, K\vardiamondsuit, A\vardiamondsuit$
\newline
\textbf{full house:} 
 a hand of 5 cards representing two cards with the same rank and another three cards with another same rank. \textit{Example:} $Q\varheartsuit, Q\spadesuit, Q\clubsuit, K\spadesuit, K\clubsuit$ 
\newline
Now, we can solve the problem itself.
\newline
There are 52 cards in a deck, 4 of each type.
The poker playeralina.abdra16@gmail.com has 5 cards, leaving 47 in the deck.
To get a full house he needs one more ace. there are 3 left in the deck.
To get a 4 of a kind he needs one more queen. there is only one left in the deck.
\newline
There are 3 aces, so three of those circumstances. there are 47 cards remaining in the deck, so there is a total of 47 circumstances. the probability of a full house is $P_{full house}=\frac{3}{47}$.
\newline
The same thing applies to the four-of-a-kind. There is only 1 queen left, so just one of those circumstances, and there are 47 cards, so 47 possible circumstances. the probability of getting a four-of-a-kind is $P_{four-of-a-kind}=\frac{1}{47}$
\newline
Thus, the final probability is: 
\newline 
P(draws to full house or four-of-a-kind)$=P_{full-house} + P_{four-of-a-kind}$
\newline
P(draws to full house or four-of-a-kind)$=\frac{3}{47}+\frac{1}{47}=\frac{4}{47}$


\subsection*{Question 5, Section 2, Chapter 2 (\cite{boyce11th})}

\noindent\textbf{Assigned to Dariya Kalymova}\\

\noindent A bridge hand (thirteen cards) is dealt from a stan dard 52-card deck. Let A be the event that the hand contains four aces; let B be the event that the hand contains four kings. Find
\begin{math}
 P(A \cup B).
 \end{math}
\newline
\textit{\textbf{Solution:}}
\begin{math}
 P(A) = \frac{\binom{4}{4} \binom{48}{9}}{\binom{52}{13}}.\newline 
  P(B) = \frac{\binom{4}{4} \binom{48}{9}}{\binom{52}{13}}.\newline
   P(A \cap B) = \frac{\binom{4}{4} \binom{4}{4} \binom{44}{5}}{\binom{52}{13}}.\newline
  P(A \cup B) = P(A) + P(B)- P(A \cap B) = \frac{4 669 920}{1512}
\end{math}

\newpage
\printbibliography
\end{document}
